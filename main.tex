\documentclass[oneside]{book}

\usepackage{amsmath, amsthm, amssymb, amsfonts}
\usepackage{thmtools}
\usepackage{graphicx}
\usepackage{setspace}
\usepackage{geometry}
\usepackage{float}
\usepackage{hyperref}
\usepackage[utf8]{inputenc}
\usepackage[english]{babel}
\usepackage{framed}
\usepackage[dvipsnames]{xcolor}
\usepackage{environ}
\usepackage{tcolorbox}
\tcbuselibrary{theorems,skins,breakable}

\setstretch{1.2}
\geometry{
    textheight=10in,
    textwidth=7.5in,
    top=0.5in,
    headheight=10pt,
    headsep=10pt,
    footskip=10pt
}

% Theorem system and user-defined commands
\input{theorems.tex}
\input{commands.tex}

\begin{document}

% Custom inline title instead of \maketitle
\begin{center}
    \LARGE \textbf{Notes on Complex Numbers}
\end{center}

\section*{
Introduction
}

\ex{
When we try to look for roots of the quadratic equation
\[
x^2 + 1 = 0
\]
we find that it has no real roots ?
}

\defn{Imaginary Unit}{
Is an imaginary number which satisfies $i^2 = -1$ i.e., $i = \sqrt{-1}$.  
}

\fact{
\[
i^{4k} = 1 \hspace{0.5em} \forall \hspace{0.5em} k \hspace{0.5em} \in \hspace{0.5em} \mathbf{Z}^{\geq 0}
\]
\[
i^{4k + 1} = i \hspace{0.5em} \forall \hspace{0.5em} k \hspace{0.5em} \in \hspace{0.5em} \mathbf{Z}^{\geq 0}
\]
\[
i^{4k +2} = -1 \hspace{0.5em} \forall \hspace{0.5em} k \hspace{0.5em} \in \hspace{0.5em} \mathbf{Z}^{\geq 0}
\]
\[
i^{4k + 3} = -i \hspace{0.5em} \forall \hspace{0.5em} k \hspace{0.5em} \in \hspace{0.5em} \mathbf{Z}^{\geq 0}
\]
}

\ex{
When we look for roots of $x^2 - 4x + 5 = 0$
\[
x = \frac{4 \pm \sqrt{16 - 4(5)}}{2} = \frac{4 \pm \sqrt{-4}}{2} = \frac{4 \pm 2.\sqrt{-1}}{2} = 2 \pm i
\]
these two roots are not purely imaginary.
}

\defn{Complex Number}{
A complex number is a number that contains 2 parts: a real and an imaginary one. 
\[
z = a + bi
\]
where a, b $\in$ $\mathbf{R}$ \& i is the imaginary unit.
\[
\mathbf{C} = \left\{ a + bi \hspace{0.5em} | \hspace{0.5em} a, b \in \mathbf{R} \text{ \& } i = \sqrt{-1} \right\}.
\]
}

\fact{
1) If a = 0 in z = a + bi, we have the set of imaginary numbers (imaginary unit . real scalars).\newline
2) If b = 0 in z = a + bi, we have $\mathbf{R}$'s\newline
\textbf{Notation}: Re(z) = a \& Im(z) = b for z = a + bi.
}

\defn{Cartesian form of a Complex Number}{A complex number is expressed in the form a + bi is said to be in Cartesian form}

Whether $ax^2 + bx + c$ has real or complex roots depends on the sign of the discriminant: $b^2 - 4ac$, real if $\geq$ 0, complex otherwise.

\section*{Arithmetic in Cartesian Form}
\subsection*{Addition}
\[
z \pm w = (a + bi) \pm (c + di)
\]
\[
z \pm w = (a \pm c) + (b \pm d)i
\]
\[
z \pm w = \left[ Re(z) \pm Re(w) \right] + \left[ Im(b) \pm Im(d) \right]i
\]

\subsection*{Multiplication}
\[
z.w = (a + bi).(c + di)
\]
\[
z.w = ac + adi + bci + bd(i^2)
\]
\[
z.w = ac + (ad + bc)i + bd(-1)
\]
\[
z.w = (ac - bd) + (ad + bc)i
\]

\subsection*{Division}
Is tricky as we not yet have a sense of what does $\frac{1}{z}$ for complex number z mean ?

\defn{Complex Conjugate}{The complex conjugate of a number z = a + bi is the complex number defined by $\overline{z}$ = a - bi}

\fact{
\mathbf{Properties of complex conjugates}
\begin{itemize}
    \item $\overline{z \pm w} = \overline{z} \pm \overline{w}$
    \item $\overline{z.w} = \overline{z}.\overline{w}$
    \item $\overline{z^{n}} = {\overline{z}}^n$
    \item If z = a + bi then $z.\overline{z} = (a + bi)(a - bi) = a^2 + b^2$.
\end{itemize}
}

\ex{Now, if w $\neq$ 0, then to find $\frac{z}{w}$ we multiply both the numerator \& denominator by the complex conjugate like so,
\[
\frac{z}{w} = \frac{z\overline{w}}{w\overline{w}} = \frac{(a + bi)(c - di)}{(c + di)(c - di)} = \frac{ac + bd + (bc - ad)i}{c^2 + d^2} = \frac{(ac + bd)}{c^2 + d^2} + \frac{(bc - ad)i}{c^2 + d^2}.
\]
}

\textbf{Note}: This process is analogous to rationalising a fraction where the denominator is a surd.

\subsection*{Equality}
Two complex numbers in the Cartesian form are equal if and only if their real and imaginary parts are equal,

\[
\text{If } z = a + bi \text{ \& } w = c + di \rightarrow z = w \iff a = c \text{ \& } b = d. 
\]

\subsection*{lack of Ordering}
Given a \& b where a $\neq$ b \& a, b $\in \mathbf{R}$ $\exists$ a trichotomy that a $>$ b or b $<$ a or a = b but only one. Unfortunately, no such statement can be made in regard to z \& w complex numbers. We can only comment on their real parts.

\section*{Complex/Argand plane}
\begin{figure}[H]
    \center
    \includegraphics[scale=0.7]{img/complexplane.png}
    \caption{the complex/Argand plane}
\end{figure}

\subsection*{Modulus}
In the context of real numbers, the modulus is the distance a point has from the origin on the real number line.
\[
|x| = 
\begin{cases}
    x \text{ if } x > 0 \\ 
    -x \text{ if } x \leq 0
\end{cases}
\]
In the case of the complex number, the modulus is defined in the same manner.

\begin{figure}[H]
    \centering
    \includegraphics[width=0.5\linewidth]{math-notes-template/img/modulus.png}
    \caption{Modulus in Complex plane}
    \label{fig:enter-label}
\end{figure}

\defn{Modulus}{For a complex number z = a + bi, $|z|$ is defined as,
\[
|z| = \sqrt{a^2 + b^2}
\]
}

\fact{$|z|^2 = z.\overline{z}$}

\fact{
\textbf{Properties of a modulus}
\begin{itemize}
    \item $|z.w| = |z|.|w|$
    \item $\left|\frac{z}{w}\right| = \frac{|z|}{|w|}$
    \item $|z + w| \leq |z| + |w|$ (The triangle inequality)
    \item $|z - w| \geq |z| - |w|$.
\end{itemize}
}

\subsection*{Subsets of the Complex Plane}

The modulus can be used to specify concentric subsets in the complex plane.

\ex{
\{z $\in$ $\mathbf{C}$ | $|z| > 2$\}

\begin{figure}[H]
    \centering
    \includegraphics[width=0.25\linewidth]{math-notes-template/img/subset1.png}
    \label{fig:enter-label}
\end{figure}
}

\ex{
\{z $\in$ $\mathbf{C}$ | $1 \leq |z| \leq 3$\}
}
\begin{figure}[H]
    \centering
    \includegraphics[width=0.30\linewidth]{math-notes-template/img/subset2.png}
    \label{fig:enter-label}
\end{figure}

\section*{Standard Polar Form}
In the cartesian system we use two axes which are perpendicular to each other. We then fix origin \& accept pairs of values which describe how far along to go on each axis. Extrapolating perpendiculars from these points and finding their intersection gives us the location.\newline

There is another way, what if you just fixed the axes \& gaves us the angle to rotate (like a bearing) \& the distance to go at that angle.
\begin{itemize}
    \item $r = |z| = \sqrt{a^2 + b^2}$
    \item $\theta = arg(z)$
\end{itemize}
Imagine the complex plane with this configuration:
\begin{figure}[H]
    \centering
    \includegraphics[width=0.35\linewidth]{math-notes-template/img/polarexplanation.png}
    \label{fig:enter-label}
\end{figure}
then,
\[
\tan{\theta} = \frac{b}{a} \text{ \& } arg(z) = \tan^{-1}{\left(\frac{b}{a}\right)}
\]
\[
a = r\cos{\theta} \text{ \& } b = r\sin{\theta}
\]
\[
\implies z = a + bi = r\cos{\theta} + r\sin{\theta}
\]

\defn{Standard Polar Form}{The standard polar form of a complex number is given by,
\[
z = r(\cos{\theta} + i\sin{\theta})
\]
where r is the modulus and $\theta$ is the argument.
}
\textbf{Note}: When trying to solve problems, remember that the symmetry of "tan()" across quadrant-1 and quadrant-3 as well as across quadrant-2 and quadrant-4 can cause problems.
\ex{
\[
z_1 = 2 + 2i \text{ is in Quadrant 1}
\]
\[
z_2 = -2 - 2i \text{ is in Quadrant 3}
\]
\[
\text{but } \tan{\left( \frac{-2}{-2} \right)} = \tan{\left(\frac{2}{2}\right)} = 1
\]
}
\ex{
Find the modulus \& argument of 3 + 7i
\[
r = \sqrt{2^2 + 7^2} = \sqrt{58}
\]
3 + 7i lies in Quadrant-1.
\begin{figure}[H]
    \centering
    \includegraphics[width=0.2\linewidth]{math-notes-template/img/example3_7i.png}
    \label{fig:enter-label}
\end{figure}
\[
\tan{\theta} = \frac{7}{3} \rightarrow \theta = \tan^{-1}{\left( \frac{7}{3}\right)}
\]
\[
\implies 3 + 7i = \sqrt{58}\left( cos\left(\tan^{-1}{\frac{7}{3}}\right) + i.\sin{\left( \tan^{-1}{\frac{7}{3}}\right)} \right)
\]
\[
\approx \sqrt{58}(cos(1.17) + i.sin(1.17))
\]
}

\ex{Write -29 in polar form\newline
\[
-29 = |29|.(cos(\pi) + i.sin(\pi)) = 29.(cos(\pi) + i.\sin(\pi))
\]}

\ex{
Convert $8\left( \cos{\left( \frac{-\pi}{6}\right)} + i.\sin{\left( \frac{-\pi}{6}\right)}\right)$ in Cartesian form.
\[
8\left( \cos{\left( \frac{-\pi}{6}\right)} + i.\sin{\left( \frac{-\pi}{6}\right)}\right) = 
8\left( \frac{\sqrt{3}}{2} + i.\frac{1}{2}.\frac{-1}{1}\right) = 4\sqrt{3} - 4i
\]
}

\textbf{Note}: $arg(z) = \theta + 2k\pi$ where $k \in \hspace{0.1em} \mathbf{Z}$, that is, we have infinitely many arguments of z.

\ex{
For z = 1 + i, we have infinitely many valid arguments
\begin{figure}[H]
    \centering
    \includegraphics[width=0.5\linewidth]{math-notes-template/img/infiniteargs.png}
    \label{fig:enter-label}
\end{figure}
}

\defn{Principal Argument}{The "principal argument" of z is denoted as Arg(z) \& is the unique argument that satisfies:
\[
-\pi < Arg(z) \leq \pi
\]
}

\thmr{Euler's Formula}{eulers}{
\[
e^{i\theta} = \cos{\theta} + i.\sin{\theta}
\]
}
Using this, 
\[
z = r(\cos{\theta} + i.\sin{\theta}) = r.e^{i\theta}
\]

\section*{Polar Exponential Form}

\defn{Polar Exponential Form}{The "Polar Exponential Form" of a complex number z = a + ib is
\[
z = re^{i\theta}
\]
where $r = \sqrt{a^2 + b^2} \text{ \& } \theta = \arg(z)$}

\subsection*{Arithmetic in Polar Exponential Form}
\subsubsection*{Multiplication}
Let $z = re^{i\theta}$ \& $w = se^{i\phi}$, then 
\[
zw = rse^{i(\theta + \phi)}
\]
\subsubsection*{Division}
Let $z = re^{i\theta}$ \& $w = se^{i\phi}$, then 
\[
\frac{z}{w} = \left(\frac{r}{s}\right)e^{i(\theta - \phi)}
\]

\thmr{}{}{
Every non-0 complex number has exactly n distinct nth complex roots.
}

\subsubsection*{Raising to an integral power}
\[
z^n = (re^{i\theta})^n = r^ne^{in\theta}
\]

\textbf{Note}: If $re^{i\theta} = se^{i\phi}$, then r = s \& $\theta = \phi + 2k\pi \hspace{0.4em} \forall \hspace{0.4em} k \in \mathbf{Z}$

\subsubsection*{Roots of complex numbers}
\[
z = e^{i\phi} \rightarrow z = e^{i(\phi + 2k\pi)}
\]
\[
z^{\frac{1}{n}} = e^{i\frac{(\phi + 2k\pi)}{n}}
\]

\textbf{Note}: The n complex roots are taken by taking n consecutive values of k.

\thmr{De Moivre's Theorem}{}{
If a complex number has modulus 1 then, 
\[
(e^{i\theta})^n = e^{in\theta}
\]
Using Euler's formula on both sides of the equation, we get
\[
(\cos{\theta} + i.\sin{\theta})^n = \cos{(n\theta)} + i\sin{(n\theta)} \hspace{0.5em} \forall \hspace{0.5em} n \in \mathbf{Z}
\]
}

\ex{
Find all the 5th roots of $-\sqrt{3} - i$, i.e., all z such that $z^5 = -\sqrt{3} - i$
\[
|-\sqrt{3} - i| = 2 \hspace{0.5em} \text{\&} \hspace{0.5em} \tan{\left ( \frac{-1}{\sqrt{3}} \right )}
\]

\begin{figure}[H]
    \centering
    \includegraphics[width=0.3\linewidth]{math-notes-template/img/sqrt3minusiota5thpowerexample.png}
    \label{fig:enter-label}
\end{figure}

\[
\therefore -\sqrt{3} - i = 2e^{-i(\frac{5\pi}{6})} = 2e^{-i(\frac{5\pi}{6} + 2k\pi)}
\]
Taking the 5th root, we get,
\[
z = 2^{1/5}e^{-i(\frac{5\pi}{6} + 2k\pi)/5} = 2^{1/5}e^{-i\left ( \frac{\pi}{6} + \frac{2k\pi}{5} \right )}
\]
For k = 0, 1, 2, 3, 4, we will obtain different values of z:
\[
\left \{ 2^{1/5}e^{-i(\frac{\pi}{6})}, 2^{1/5}e^{-i(\frac{17\pi}{30})}, 2^{1/5}e^{-i(\frac{29\pi}{30})}, 2^{1/5}e^{-i(\frac{41\pi}{30})}, 2^{1/5}e^{-i(\frac{53\pi}{30})}   \right \}
\]
}

\subsection*{Complex Roots of Real Numbers}
\ex{
Find the 4th roots of 16
\[
2^4 = (-2)^4 = 16
\]\newline
-2 and 2 are 4th roots, but,
\[
16 = 16e^{i(0 + 2k\pi)} \implies z = 16^{1/4}e^{\frac{i(2k\pi)}{4}} = 2e^{\frac{k\pi}{2}}, k  \hspace{0.5em} \in \hspace{0.5em} \mathbf{Z}
\]
\newline
\[
k = 0: z = 2
\]\newline
\[
k = 1: z = 2i
\]\newline
\[
k = 2: z = -2
\]\newline
\[
k = 3: z = -2i
\]\newline
$\therefore$ 16 has 2 real and 4th roots but 4 complex 4th roots.
}

\section*{Polynomials}
\defn{Polynomial Expression}{A polynomial in z is an expression of the form
\[
a_nz^n + a_{n-1}z^{n-1} + a_{n-2}z^{n-2} + ... + a_2z^2 + a_1z^1 + a_0z^0
\]
where z is a variable and $a_n, a_{n-1}, a_{n-2}, ... , a_2, a_ 1, a_0$ are coefficients.
}


\defn{Polynomial Equation}{A polynomial expression set to 0, we obtain a polynomial equation:
\[
a_nz^n + a_{n-1}z^{n-1} + a_{n-2}z^{n-2} + ... + a_2z^2 + a_1z^1 + a_0z^0 = 0
\]
where z is a variable and $a_n, a_{n-1}, a_{n-2}, ... , a_2, a_ 1, a_0$ are coefficients.
}

\textbf{Note}: If $a_n \neq 0$, then polynomial is said to have a degree n.

\defn{Roots of Polynomial Equation}{The roots of the polynomial equation are the numbers z, which satisfy
\[
a_nz^n + a_{n-1}z^{n-1} + a_{n-2}z^{n-2} + ... + a_2z^2 + a_1z^1 + a_0z^0 = 0,
\]
A polynomial equation of degree n has at most n complex roots.
}

\textbf{Note}: the previous example of finding 4th roots can be rephrased as finding the roots of the polynomial equation $z^4 - 16 = 0$.\newline

\textbf{Note}: The same root may occur multiple times, that is, with multiplicity greater than 1.

\thmr{}{}{
If the coefficients in a polynomial equation are all real, then all the complex roots occur in complex conjugate pairs.
}
\begin{proof}
Suppose v is a complex root of p(x), then p(v) = 0
\[
p(v) = 0
\]
\[
\overline{p(v)} = \overline{0}
\]
\[
 \overline{a_nv^n + a_{n-1}v^{n-1} + ... + a_1v^1 + a_0} = 0
\]
\[
\overline{a_nv^n} + \overline{a_{n-1}v^{n-1}} + ... + \overline{a_1v^1} + \overline{a_0} = 0
\]
\[
0 = a_n\overline{v^n} + a_{n-1}\overline{v^{n-1}} + ... + a_{1}\overline{v^{1}} + a_0 = 0
\]
\[
 a_n(\overline{v}^n) + a_{n-1}(\overline{v}^{n-1}) + ... + a_1(\overline{v}^1) + a_0 = 0
\]
\[
 p(\overline{v}) = 0
\]
\end{proof}

\section*{Sin and Cos in Terms of exponentials}
\fact{
\[
\cos{\theta} = \frac{e^{i\theta} + e^{-i\theta}}{2}
\]
\[
\sin{\theta} = \frac{e^{i\theta} - e^{-i\theta}}{2}
\]
$\forall \hspace{0.5em} \theta \hspace{0.5em} \in \hspace{0.5em} \mathbf{R}$
}

\section*{Miscellaneous}
\defn{Complex Exponential Function}{
let z = x + yi $\in \mathbf{C}$, x, y $\in \mathbf{R}$, the Complex Exponential Function is defined as,
\[
e^z = e^{x + iy} = e^xe^{iy} = e^x(\cos{y} + i.\sin{y})
\]
}
\textbf{Note}: When z = x + iy, with x, y $\in \mathbf{R}$; $|e^z| = e^x$ and arg($e^z$) = y 
\thmr{A Theorem of Remarkable Beauty}{}{
For $\theta = \pi$, Euler's identity becomes,
\[
e^{i\pi} + 1 = 0.
\]
This theorem relates the 5 most important constants in mathematics: $\pi$, e, $\theta$, 1 and 0.
}
\end{document}